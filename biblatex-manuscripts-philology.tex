\documentclass{ltxdockit}[2011/03/25]
\usepackage{btxdockit}
\usepackage{fontspec}
\usepackage[mono=false]{libertine}
\usepackage{microtype}
\usepackage[american]{babel}
\usepackage[strict]{csquotes}
\setmonofont[Scale=MatchLowercase]{DejaVu Sans Mono}
\usepackage{shortvrb}
\usepackage{pifont}
\usepackage{minted}
% Usefull commands
\newcommand{\biblatex}{biblatex\xspace}
\pretocmd{\bibfield}{\sloppy}{}{}
\pretocmd{\bibtype}{\sloppy}{}{}
% Meta-datas
\titlepage{%
	title={Manuscripts description for philology with biblatex},
	subtitle={New data types},
	email={maieul <at> maieul <dot> net},
	author={Maïeul Rouquette},
	revision={1.0.0-beta},
	date={25/11/2013},
	url={https://github.com/maieul/biblatex-manuscripts-philology}}
	

\begin{document}

\printtitlepage

\section{Introduction}
\subsection{Goals}
The \biblatex package defines some standard field for entry, and allows to use extra fields, like \bibfield{usera}, \bibfield{userb}. However, the number of extra fields needed to describe manuscripts in an introduction of a critical edition of classical texts is too great to use these field names without mistake.

The aim of this package is double:

\begin{itemize}
\item Provides new datatypes, mainly \bibtype{manuscripts} and \bibtype{inmanuscripts}, with adapted field
\item Provides new bibliography styles to print the list of manuscripts :
\begin{itemize}
	\item As a detailed list of witnesses of a text.
	\item As a \emph{conspectus siglorum}.
\end{itemize}
\end{itemize}

\subsection{Credits}

This package was created for Maïeul Rouquette's PHD\footnote{\url{http://apocryphes.hypothese.org}} in 2013. It is licenced on the \emph{\LaTeX\ Project Public Licence}\footnote{\url{http://latex-project.org/lppl/lppl-1-3c.html}.}.

All issues can be submitted, in French or English, in the GitHub issues page\footnote{\url{https://github.com/maieul/biblatex-manuscripts-philology/issues}.}.


\section{New type and fields}

This package defines one new bibtype \bibtype{manuscript}, which is to be used to defined a manuscript.

This bibtype has these mandatory fields:

\begin{fieldlist}

\fielditem{collection}{literal} the collection in the library. For example: \verb+Supplément grec+.

\fielditem{location}{literal} the city or place where the manuscript is kept. For example: \verb+Paris+ or \verb+Oxford+. 

\fielditem{library}{literal} the library where the manuscript is kept. For example: \verb+Bibliothèque Nationale de France+.


\fielditem{shelfmark}{literal} the shelfmark in the collection. For example: \verb+241+.
\end{fieldlist}

This bibtype can use the optional fields:

\begin{fieldlist}

\fielditem{bookpagination}{key} the pagination of the manuscript which is studied. The standard pagination key are allowed, but the package add a new key: \texttt{folio}. The value of this field is to be used for printing the \bibfield{pages} and \bibfield{pagetotal}.

\fielditem{columns}{integer} the number of column by pages. Basically, only two numbers are allowed : \verb+1+ or \verb+2+.

\fielditem{dating}{litteral} the dating of the manuscript. It can be for example a century.

\fielditem{origin}{litteral} the place(s) where the manuscript was written.

\fielditem{pages}{range} the pages which are studied in the manuscript. In this field, you can use the macros \cs{recto} and \cs{verso}.

\fielditem{pagetotal}{integer} the total number of page in the manuscript.


\fielditem{pagination}{key} the pagination of part of manuscript which is studied. The standard pagination key are allowed, but the package add a new key: \verb+folio+.

\fielditem{scribe}{name} the scribe(s) who wrote the manuscript.

\fielditem{shorthand}{litteral} the shorthand of the manuscript. If this field is empty, the key will be used as shorthand.

\fielditem{support}{key} the support of the manuscript, which will be translated in your work language. These keys are defined : \texttt{papyrus}, \texttt{paper}, \texttt{pergament}.

\end{fieldlist}

There are also two special fields, which are printed only if we ask for them explicitly: 

\begin{fieldlist}
\fielditem{Annotation}{litteral} some annotation about the manuscript and its content. If you want to add paragraph inside it, you must use the \cs{par} command between each paragraph.

\fielditem{catalog}{special} a list of catalogues which describe the manuscript. You must fill this field with arguments of a \cs{cites} command. For example:

\begin{minted}{tex}
@manuscript{key,
  field1 = {value1},
  field2 = {value2},
  catalog = {[prenote1][postenote1]{key1}[prenote2][postenote2]{key2}}
\end{minted}
\end{fieldlist}


\section{Use}
\subsection{Loading}

When loading the \biblatex package, use the option \opt{bibstyle} with value equal to \opt{manuscripts}.

\begin{minted}{latex}
\usepackage[bibstyle=manuscripts,…]{biblatex}
\end{minted}
\end{document}