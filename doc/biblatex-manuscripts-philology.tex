\documentclass{ltxdockit}[2011/03/25]
\usepackage{btxdockit}
\usepackage{fontspec}
\usepackage[mono=false]{libertine}
\usepackage{microtype}
\usepackage[american]{babel}
\usepackage[strict]{csquotes}
\setmonofont[Scale=MatchLowercase]{DejaVu Sans Mono}
\usepackage{shortvrb}
\usepackage{pifont}
% Usefull commands
\newcommand{\biblatex}{biblatex\xspace}
\pretocmd{\bibfield}{\sloppy}{}{}
\pretocmd{\bibtype}{\sloppy}{}{}
% Meta-datas
\titlepage{%
	title={Manuscripts description for philology with biblatex},
	subtitle={New data types},
	email={maieul <at> maieul <dot> net},
	author={Maïeul Rouquette},
	revision={1.0.0-beta},
	date={25/11/2013},
	url={https://github.com/maieul/biblatex-manuscripts-philology}}
	

\begin{document}

\printtitlepage

\section{Introduction}
\subsection{Goals}
The \biblatex package defines some standard field for entry, and allows to use extra fields, like \bibfield{usera}, \bibfield{userb}. However, the number of extra fields needed to describe manuscripts in an introduction of a critical edition of classical texts is too great to use these field names without mistake.

The aim of this package is double:

\begin{itemize}
\item Provides new datatypes, mainly \bibtype{manuscripts} and \bibtype{inmanuscripts}, with adapted field
\item Provides new bibliography styles to print the list of manuscripts :
\begin{itemize}
	\item As a detailed list of witnesses of a text.
	\item As a \emph{conspectus siglorum}.
\end{itemize}
\end{itemize}

\subsection{Credits}

This package was created for Maïeul Rouquette's PHD\footnote{\url{http://apocryphes.hypothese.org}} in 2013. It is licenced on the \emph{\LaTeX\ Project Public Licence}\footnote{\url{http://latex-project.org/lppl/lppl-1-3c.html}.}.

All issues can be submitted, in French or English, in the GitHub issues page\footnote{\url{https://github.com/maieul/biblatex-manuscripts-philology/issues}.}.


\end{document}